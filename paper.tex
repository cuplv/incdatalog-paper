%% For double-blind review submission
\documentclass[acmlarge,anonymous]{acmart}\settopmatter{printfolios=true}
%% For single-blind review submission
%\documentclass[acmlarge,review]{acmart}\settopmatter{printfolios=true}
%% For final camera-ready submission
%\documentclass[acmlarge]{acmart}\settopmatter{}

%% Note: Authors migrating a paper from PACMPL format to traditional
%% SIGPLAN proceedings format should change 'acmlarge' to
%% 'sigplan,10pt'.


%% Some recommended packages.
\usepackage{booktabs}   %% For formal tables:
                        %% http://ctan.org/pkg/booktabs
\usepackage{subcaption} %% For complex figures with subfigures/subcaptions
                        %% http://ctan.org/pkg/subcaption

\usepackage{listings}

\makeatletter\if@ACM@journal\makeatother
%% Journal information (used by PACMPL format)
%% Supplied to authors by publisher for camera-ready submission
\acmJournal{PACMPL}
\acmVolume{1}
\acmNumber{1}
\acmArticle{1}
\acmYear{2017}
\acmMonth{1}
\acmDOI{10.1145/nnnnnnn.nnnnnnn}
\startPage{1}
\else\makeatother
%% Conference information (used by SIGPLAN proceedings format)
%% Supplied to authors by publisher for camera-ready submission
\acmConference[PL'17]{ACM SIGPLAN Conference on Programming Languages}{January 01--03, 2017}{New York, NY, USA}
\acmYear{2017}
\acmISBN{978-x-xxxx-xxxx-x/YY/MM}
\acmDOI{10.1145/nnnnnnn.nnnnnnn}
\startPage{1}
\fi


%% Copyright information
%% Supplied to authors (based on authors' rights management selection;
%% see authors.acm.org) by publisher for camera-ready submission
\setcopyright{none}             %% For review submission
%\setcopyright{acmcopyright}
%\setcopyright{acmlicensed}
%\setcopyright{rightsretained}
%\copyrightyear{2017}           %% If different from \acmYear


%% Bibliography style
\bibliographystyle{ACM-Reference-Format}
%% Citation style
%% Note: author/year citations are required for papers published as an
%% issue of PACMPL.
\citestyle{acmauthoryear}   %% For author/year citations



\begin{document}

%% Title information
\title[Short Title]{Patchwork: Incremental Computation for Static Analysis}         %% [Short Title] is optional;
                                        %% when present, will be used in
                                        %% header instead of Full Title.
\titlenote{with title note}             %% \titlenote is optional;
                                        %% can be repeated if necessary;
                                        %% contents suppressed with 'anonymous'
\subtitle{Subtitle}                     %% \subtitle is optional
\subtitlenote{with subtitle note}       %% \subtitlenote is optional;
                                        %% can be repeated if necessary;
                                        %% contents suppressed with 'anonymous'


%% Author information
%% Contents and number of authors suppressed with 'anonymous'.
%% Each author should be introduced by \author, followed by
%% \authornote (optional), \orcid (optional), \affiliation, and
%% \email.
%% An author may have multiple affiliations and/or emails; repeat the
%% appropriate command.
%% Many elements are not rendered, but should be provided for metadata
%% extraction tools.

%% Author with single affiliation.
\author{First1 Last1}
\authornote{with author1 note}          %% \authornote is optional;
                                        %% can be repeated if necessary
\orcid{nnnn-nnnn-nnnn-nnnn}             %% \orcid is optional
\affiliation{
  \position{Position1}
  \department{Department1}              %% \department is recommended
  \institution{Institution1}            %% \institution is required
  \streetaddress{Street1 Address1}
  \city{City1}
  \state{State1}
  \postcode{Post-Code1}
  \country{Country1}
}
\email{first1.last1@inst1.edu}          %% \email is recommended

%% Author with two affiliations and emails.
\author{First2 Last2}
\authornote{with author2 note}          %% \authornote is optional;
                                        %% can be repeated if necessary
\orcid{nnnn-nnnn-nnnn-nnnn}             %% \orcid is optional
\affiliation{
  \position{Position2a}
  \department{Department2a}             %% \department is recommended
  \institution{Institution2a}           %% \institution is required
  \streetaddress{Street2a Address2a}
  \city{City2a}
  \state{State2a}
  \postcode{Post-Code2a}
  \country{Country2a}
}
\email{first2.last2@inst2a.com}         %% \email is recommended
\affiliation{
  \position{Position2b}
  \department{Department2b}             %% \department is recommended
  \institution{Institution2b}           %% \institution is required
  \streetaddress{Street3b Address2b}
  \city{City2b}
  \state{State2b}
  \postcode{Post-Code2b}
  \country{Country2b}
}
\email{first2.last2@inst2b.org}         %% \email is recommended


%% Paper note
%% The \thanks command may be used to create a "paper note" ---
%% similar to a title note or an author note, but not explicitly
%% associated with a particular element.  It will appear immediately
%% above the permission/copyright statement.
\thanks{with paper note}                %% \thanks is optional
                                        %% can be repeated if necesary
                                        %% contents suppressed with 'anonymous'


%% Abstract
%% Note: \begin{abstract}...\end{abstract} environment must come
%% before \maketitle command
\begin{abstract}
Text of abstract \ldots.
\end{abstract}


%% 2012 ACM Computing Classification System (CSS) concepts
%% Generate at 'http://dl.acm.org/ccs/ccs.cfm'.
\begin{CCSXML}
<ccs2012>
<concept>
<concept_id>10011007.10011006.10011008</concept_id>
<concept_desc>Software and its engineering~General programming languages</concept_desc>
<concept_significance>500</concept_significance>
</concept>
<concept>
<concept_id>10003456.10003457.10003521.10003525</concept_id>
<concept_desc>Social and professional topics~History of programming languages</concept_desc>
<concept_significance>300</concept_significance>
</concept>
</ccs2012>
\end{CCSXML}

\ccsdesc[500]{Software and its engineering~General programming languages}
\ccsdesc[300]{Social and professional topics~History of programming languages}
%% End of generated code


%% Keywords
%% comma separated list
\keywords{keyword1, keyword2, keyword3}  %% \keywords is optional


%% \maketitle
%% Note: \maketitle command must come after title commands, author
%% commands, abstract environment, Computing Classification System
%% environment and commands, and keywords command.
\maketitle


\section{Introduction}

Programmers who use static analyses in their normal workflow will frequently demand the results of these analyses multiple times. For example, when working on code in an IDE, a programmer may make multiple small edits and request the results of an anlysis after each one. Since these edits are likely to be minor, performing the entire analysis from scratch each time is inefficient. This is a prime application for \textit{incremental computation} (insert typical IC story). 

Incremental computation is reasonably well understood for standard Datalog, which can express a small class of analyses. However, more expressiveness is needed to implement complex lattice-based analyses, particularly on infinite-height lattices, such as an interval analysis. We present the Patchwork framework, which extends the Datalog-based approach to allow for incremental implementations of general static analyses.

%\newcommand{\TabYes}{\ensuremath{\ding{52}}}
%\newcommand{\TabNo}{\ensuremath{\ding{53}}}
\newcommand{\TabYes}{\ensuremath{\blacksquare}}
\newcommand{\TabNo}{\ensuremath{\square}}

\begin{table}
\begin{tabular}{|p{1.3in}p{1.5in}||ccc|l|}
  \hline
  \textbf{program domain} &
  \textbf{system feature} & Flix & IncA & Datafun & \emph{this paper}
  \\
  \hline
  general &
  first-class  
  fixed points & \TabNo & \TabNo & \TabYes & \TabYes
  \\
  general &
  user-defined
  datatypes & \TabYes & ? & \TabYes & \TabYes
  \\
  \hline
  datalog &
  datalog engine & \TabYes & \TabYes & \TabYes & \TabYes~(definable)
  \\
  \hline
  program analysis &
  user-defined lattices & \TabYes & \TabYes & \TabNo & \TabYes
  \\
  program analysis &
  non-monotone 
  widening
  & \TabYes & \TabNo & n/a & \TabYes
  \\
  program analysis &
  relational 
  analysis
  & ? & \TabNo? & n/a & \TabYes
  \\
  \hline
\end{tabular}
\caption{By-feature comparison with related work on datalog semantics, and (incremental) datalog for program analysis.}
\label{tab:relatedwork}
\end{table}

\section{Motivating Examples}

Parity analysis (requires lattice)\\
Interval analysis (requires widening)

\section{Approach: fungilang + lattice?}

Present fungilang semantics

\section{Worked examples}

go back through motivating examples: parity analysis

\begin{lstlisting}
1 x = 1;
2 while x < rand()
3   x++;
\end{lstlisting}

\section{Evaluation}

We evaluated our approach on a spread of analyses and target programs.

Analyses:\\
Simple interval? Points-to? Interprocedural?

Programs:\\
IncA benchmarks? Flix benchmarks? Both?

Edit model:\\
Adopt IncA approach? Commit history? Other?

We compare to {Flix/IncA}?

\section{Future work/Extensions}

Interprocedural? Or doing that now?

\section{Related Work}

Incremental Datalog framework (which?):

Flix: the Patchwork framework extends the work of the Flix authors. Specifically, Flix provides a Datalog extension with the ability to reason about lattices. Patchwork extends work done on incremental Datalog evaluation, to, effectively, incremental Flix evaluation. Notably, Flix evaluates (large numbers) of rules in the course of (example). This includes significant redundancy, due to (facts about their dependency analysis). Our approach automatically tracks these dependencies and improves performance by (numbers).

IncA: the IncA papers gives an algorithm for performing incremental evaluation of lattice-aware Datalog-style programs. The approach takes advantage of properties of monotonic operators on lattices to quickly update aggregations and Datalog relations. However, the monotonicity restriction rules out important program analyses, such as (those that depend on widening). The IncA framework cannot express our second worked example. Additionally, (performance).

\section{Conclusion}

Existing state-of-the-art approaches to incremental program analysis are not fully general. Incremental Datalog engines are restricted to the simple declarative model, and even more complex approaches such as IncA restrict the types of operators an analysis user can work with. The Patchwork framework provides a fully general (is this true?) approach to incremental static analysis. Our use of Adapton also improves efficiency by (numbers). This represents a significant advance in the state of incremental static analysis.

%% Acknowledgments
\begin{acks}                            %% acks environment is optional
                                        %% contents suppressed with 'anonymous'
  %% Commands \grantsponsor{<sponsorID>}{<name>}{<url>} and
  %% \grantnum[<url>]{<sponsorID>}{<number>} should be used to
  %% acknowledge financial support and will be used by metadata
  %% extraction tools.
  This material is based upon work supported by the
  \grantsponsor{GS100000001}{National Science
    Foundation}{http://dx.doi.org/10.13039/100000001} under Grant
  No.~\grantnum{GS100000001}{nnnnnnn} and Grant
  No.~\grantnum{GS100000001}{mmmmmmm}.  Any opinions, findings, and
  conclusions or recommendations expressed in this material are those
  of the author and do not necessarily reflect the views of the
  National Science Foundation.
\end{acks}


%% Bibliography
%\bibliography{bibfile}


%% Appendix
\appendix
\section{Appendix}

Text of appendix \ldots

\end{document}
